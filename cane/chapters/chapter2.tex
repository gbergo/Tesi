%!TEX root = ../dissertation.tex

\hypertarget{(chap:capitolo2)}{}
\chapter{Processi e metodologie}
In questo capitolo verranno riportati in modo approfondito lo scopo e gli obiettivi dello stage, contestualizzazione delle attività alla realtà aziendale e scadenzario delle stesse.

\section{Contesto}
Il progetto generale nasce dalla visione dell'azienda, di studiare e implementare moduli sulla piattaforma ERP open Odoo.

Il software è \glo{open source} e si può considerare una suite di prodotti software attivamente supportato da una community internazionale ed una italiana ai fini della localizzazione.
Alcune caratteristiche che permettono di soddisfare gli obbiettivi sono: contabilità finanziaria, contabilità analitica, gestione del magazzino, gestione di vendite e acquisti, automazione dei processi, gestione risorse umane. (altre caratteristiche delle piattaforma:campagne di marketing, siti web aziendali e il software e-commerce).

\section{Introduzione al progetto}
Il progetto di stage, dopo l'evento \bit{StageIt}{stageit} ed alcune riunioni presso l'azienda, è stato studiato dettagliatamente e riportato nel documento "piano di lavoro", nello stesso sono riportati gli obiettivi e la pianificazione delle attività.

Questi esempi fanno riferimento a problematiche in ambito produttivo "Food and beverage",, realizzando un modulo che si occuperà di monitorare il ciclo di vita degli alimenti e delle bevande il tutto per ottimizzare il cosiddetto Time-to-Market, mantenendo ottimali i livelli di inventario e ottimizzando i flussi economici dei prodotti.
Lo stage prevede la realizzazione di diversi moduli il cui sviluppo sarà incrementale in quanto ciascun modulo eredita struttura del modello e viste dai precedenti.

A rendere più complicato il lavoro dell'apprendimento di odoo riportiamo alcuni principali problemi da tenere in considerazione:
\begin{itemize}
	\item Gestione del database: si appoggia a \bit{pgAdmin 4}{pgadmin}, che consente di gestire in modo semplificato database PostgreSQL, verificando il corretto inserimento o eliminazione di dati;
	\item Gestire gli errori: organizzare una buona vista tramite i tag odoo, XML e XPath idonei, in modo da controllare gli errori e avvisi.
\end{itemize}

\section{Vincoli temporali, tecnologici e metodologici}
Nel periodo di stage svolto presso l'azienda mi è stato chiesto di svolgere dei task e tenere traccia del loro stato di completamento, tramite un applicazione via web condivisa, utilizzando \bit{OpenProject}{openproject}, nel suddetto mi si richiedeva di annotare giornalmente l'avanzare del lavoro riportando idee e osservazioni emerse durante i \glo{brainstorming} quotidiani, criticità rilevate e positività riscontrate negli strumenti utilizzati.
Le telefonate con i ragazzi delle sede di Napoli, che seguivano il progetto, sono state molto importanti per chiarire le consegne date su ogni task e la sintassi corretta da seguire.

Prima di iniziare lo stage è stato concordato con l'azienda un piano di lavoro su un totale di 320 ore, lavorando 5 giorni a settimana, 8 ore per ciascun giorno. 

\section{Requisiti e obiettivi}
Nell'elenco di seguito vengono riportati gli obiettivi dello stage, a corredo degli stessi vi sarà un codice univoco ed una breve descrizione.

Ogni obiettivo è provvisto di un codice identificativo formato da una delle seguenti stringhe ob,de,op, che rappresentano il livello di importanza e da un numero incrementale positivo, che rispetta la seguente nomenclatura: 
\begin{figure}[htp]
	\centering
	[importanza][identificativo].
\end{figure}

Il livello di importanza di ciascun obiettivo può essere uno tra i seguenti:
\begin{itemize}
	\item Obbligatorio: individuato dalla stringa \textit{ob}, sono obiettivi fondamentali per la riuscita del progetto, il loro soddisfacimento dovrà verificarsi obbligatoriamente entro la fine dello stage, pena il fallimento dello stesso;
	\item Desiderabile: individuato dalla stringa \textit{de}, sono obiettivi secondari su cui però si nutre dell'interesse, il loro soddisfacimento è auspicabile entro la fine dello stage;
	\item Opzionale: individuato dalla stringa \textit{op}, sono obiettivi di contorno su cui si nutre poco interesse, la loro realizzazione si verificherà nel momento in cui si dovessero soddisfare tutti gli obiettivi obbligatori e desiderabili prima della fine dello stage.
\end{itemize}


Di seguito riportiamo la lista degli obbiettivi pianificati:
\begin{itemize}
	\item Obbligatori
	      \begin{itemize}
	      	\item \underline{\textit{ob01}}: Implementazione di un modulo che si occuperà di monitorare il ciclo di vita degli alimenti e delle bevande con l’obiettivo di ottimizzare il cosiddetto Time-to-Market, mantenendo ottimali i livelli di inventario ed ottimizzando i flussi economici dei prodotti;
	      	\item \underline{\textit{ob02}}: Acquisizione competenze sulle tematiche sopra descritte;
	      	\item \underline{\textit{ob03}}: Capacità di raggiungere gli obbiettivi richiesti in autonomia seguendo il programma;
	      	\item \underline{\textit{ob04}}: Portare a termine le modifiche richieste dal cliente con una percentuale di superamento degli item di collaudo pari al 50%.
	      \end{itemize}
	\item Desiderabili
	      \begin{itemize}
	      	\item \underline{\textit{de01}}: Acquisizione di un buon livello di autonomia sulla piattaforma Erp Odoo;
	      	\item \underline{\textit{de02}}: Portare a termine le modifiche richieste dal cliente con una percentuale di superamento degli item di collaudo pari all’80\%.
	    
	      \end{itemize}
	\item Opzionali
	      \begin{itemize}
	      	\item \underline{\textit{op01}}: Implementazione di nuovi moduli.
	      \end{itemize} 
\end{itemize}

\section{Pianificazione}
Con le ore a disposizione per questo stage si è proceduto a organizzare come segue le attività:
\begin{itemize}
	\item \textbf{Formazione}: si è visto necessario approfondire i database \bit{PostgreSql}{postgresql}, il linguaggio di markup XML-XPath, Html5/CSS3, il linguaggio di scripting orientato agli oggetti e agli eventi \bit{Javascript}{javascript}, imparare il linguaggio di programmazione \bit{Python}{python} e lo studio della Piattaforma Erp \bit{Odoo}{odoo}
	\item \textbf{Sviluppo moduli Odoo}: Implementazione nuovi moduli in ambito Food-and-Beverage o modifiche su esistenti
	\item \textbf{Collaudo Finale}: Esecuzione dei test e collaudo
\end{itemize}

\newpage
La pianificazione delle attività è stata distribuita come mostrato nella Tabella \ref{tableofwork}.
\begin{center}
	\begin{tabular}{|l|l|c l|}
		\hline
		\multicolumn{2}{|l|}{\textbf{Durata in ore}}		&	\multicolumn{2}{l|}{\textbf{Descrizione dell'attività}}\\
		\hline
		\multicolumn{2}{|l|}{208}	&	\multicolumn{2}{l|}{\textbf{A}: Formazione}\\
		\hline
		\multirow{5}{1cm}{ } & 15  & \hspace{5mm}•\hspace{2mm} & Studio Database Sql-PostgreSql \\
		\multirow{5}{1cm}{ } & 15 & \hspace{5mm}•\hspace{2mm} & Studio XML-XPath                 \\
		\multirow{5}{1cm}{ } & 10 & \hspace{5mm}•\hspace{2mm} & Studio Html5/CSS3\\
		\multirow{5}{1cm}{ } & 20 & \hspace{5mm}•\hspace{2mm} & Studio Javascript\\
		\multirow{5}{1cm}{ } & 53 & \hspace{5mm}•\hspace{2mm} & Studio linguaggio Python\\
		\multirow{5}{1cm}{ } & 95 & \hspace{5mm}•\hspace{2mm} & Studio Piattaforma Erp Odoo\\
		\hline																											
		\multicolumn{2}{|l|}{72}	&	\multicolumn{2}{l|}{\textbf{B}: Sviluppo moduli}\\
		\hline
		\multirow{5}{1cm}{ } & 72  & \hspace{5mm}•\hspace{2mm} & Implementazione di nuovi moduli ODOO  \\ & & & o modifiche su esistenti\\
		
		\hline
																											
		\multicolumn{2}{|l|}{40}	&	\multicolumn{2}{l|}{\textbf{C}: Collaudo Finale}\\
		\hline
		\multirow{5}{1cm}{ } & 20  & \hspace{5mm}•\hspace{2mm} & Test\\
		\multirow{5}{1cm}{ } & 15 & \hspace{5mm}•\hspace{2mm} & Stesura Documentazione Finale \\
		\multirow{5}{1cm}{ } & 5 & \hspace{5mm}•\hspace{2mm} & Collaudo e consegna del codice\\

		\hline
																	
		\multicolumn{2}{|l|}{\textbf{Totale: 320}} & \multicolumn{2}{l|}{}\\
		\hline
																												
	\end{tabular}
	\captionof{table}{Pianificazione concordata nel piano di lavoro}
	\label{tableofwork}  
\end{center}
\newpage
\section{Ambiente di lavoro}
\subsection{Metodi di sviluppo}
Il \glo{ciclo di vita} di un prodotto Sync Lab segue il \glo{modello evolutivo}, formato da un periodo di concezione dell'idea, analisi della stessa, realizzazione di una singola evoluzione ed infine, si rilascia il prodotto, che può avere più versioni esterne attive in parallelo e dovrà mostrare le nuove funzionalità sviluppate dimostrando così l'incremento fatto. In quest'ottica, lo sviluppo di un modulo può essere associato ad un iterazione, la cui \glo{milestone} è il passaggio dei test.


\subsection{Gestione di progetto}
Per quanto riguarda la gestione di progetto sono stati utilizzati alcuni strumenti descritti con maggiore dettaglio nel \hyperlink{(chap:capitolo6)}{\textbf{Capitolo 6}}. In generale per la gestione dei task da eseguire si è fatto uso di \bit{OpenProject}{openproject}, uno strumento di \glo{project management}; per la gestione della comunicazione e informazioni si è fatto uso dell'applicazione \bit{Slack}{slack} e per la condivisione di \bit{documentazione}{doc} e articoli si è fatto uso della e-mail.
	
\subsection{Linguaggio di programmazione e ambiente di sviluppo}
Per la totalità dello stage si è lavorato utilizzando 
\bit{PyCharm}{pycharm}. Con questo strumento è stato possibile scrivere il modello in linguaggio \bit{Python}{python} in modo molto agevole. 

Questo linguaggio di programmazione è orientato agli oggetti ed interpretato dinamicamente al momento dell'esecuzione da un interprete. \bit{Python}{python} risulta veramente versatile in quanto fornisce incredibili funzionalità utilizzabili in modo semplice e intuitivo, dispone di moltissimi moduli che permettono le più svariate operazioni.
\newpage

\section{Analisi dei rischi}
In questa sezione vengono riportati i principali rischi che si prefiguravano all'inizio dello stage. Ciascuno di essi oltre ad avere una breve descrizione riporta il livello di rischio, in termini di pericolosità per la riuscita dello stage e come si possa fare per evitarlo:
\begin{itemize}
	\item \textbf{Difficoltà nelle tecnologie adottate}\\
	      Ad inizio stage è stato chiaro che \bit{Python}{python} avrebbe avuto un ruolo dominante nel progetto, la mole di librerie ed estensioni rendeva il linguaggio troppo vasto da poter approfondire nella sua interezza, ed oltre a questo vi era le views da sviluppare
	      \begin{itemize}
	      	\item \textbf{Livello di rischio}: Medio;
	      	\item \textbf{Contromisure}: Studiare approfonditamente la costruzione di un modello Odoo tramite esercitazione.
	      \end{itemize}
	\item \textbf{Difficoltà di integrazione nel team}\\
	      Di fondamentale importanza per la riuscita di un progetto è la cooperazione con i colleghi e la creazione di un ambiente di lavoro sano e che stimoli la produttività, essendo un nuovo arrivato inserito in un ambiente soggetto a forte stress per le stringenti scadenze vi era la possibilità di entrare in conflitto con qualche collega.
	      \begin{itemize}
	      	\item \textbf{Livello di rischio}: Basso;
	      	\item \textbf{Contromisure}: Perseguire un atteggiamento positivo, critico e oggettivo.
	      \end{itemize}
	\item \textbf{Contrattempi dovuti a malattie e impegni}\\
	      Un rischio da tenere in considerazione è quello dovuto a impegni o malattie che precludano la possibilità di recarsi nel luogo di lavoro, data la durata dello stage è sicuramente possibile possa verificarsi.
	      \begin{itemize}
	      	\item \textbf{Livello di rischio}: Basso;
	      	\item \textbf{Contromisure}: Organizzare precedentemente ogni impegno non lavorativo e tempo di \glo{slack} per evitare contrattempi.
	      \end{itemize}
	\item \textbf{Difficoltà di stima dei tempi previsti}\\
	      Con un progetto di così lunga durata e l'inesperienza che ci portiamo appresso è possibile che vengano fatti degli errori di valutazione in termini di tempistiche per lo svolgimento delle diverse attività pianificate.
	      \begin{itemize}
	      	\item \textbf{Livello di rischio}: Medio;
	      	\item \textbf{Contromisure}: Rendere partecipi nella definizione del piano di lavoro persone esperte, come il tutor aziendale e il project manager.
	      \end{itemize}
\end{itemize}
