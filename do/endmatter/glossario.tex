%!TEX root = ../dissertation.tex

\begin{itemize} 

	\item \gloTarget{brainstorming} incontri di gruppo creativi utili per risolvere problemi o individuare nuove idee. Servono più di due partecipanti in modo che vi sia una discussione arbitraria e quindi utile.
	\item \gloTarget{ciclo di vita} insieme degli stati che il prodotto assume dal concepimento al ritiro.
	\item \gloTarget{Food-and-beverage} settore riguardante la ristorazione, alimentare cioè cibi e bevande
	\item  \gloTarget{Incrementale} si procede per aggiunte ad un impianto base
	\item \gloTarget{modello evolutivo} modello di ciclo di vita che aiuta a rispondere a bisogni non inizialmente preventivabili, può richiedere il rilascio di più versioni esterne attive in parallelo. La grande differenza con il modello incrementale è che l'analisi e progettazione iniziali vengono ripetute.
	\item \gloTarget{milestone} punto nel tempo al quale associamo un insieme di stati di avanzamento.
	\item \gloTarget{open source} termine utilizzato per riferirsi ad un software di cui i detentori dei diritti sullo stesso ne rendono pubblico il codice sorgente.
	\item \gloTarget{project management} gestione delle attività di analisi, progettazione, pianificazione e realizzazione degli obiettivi di un progetto, compito svolto dal project manager di un'azienda attraverso anche strumenti idonei.
	\item \gloTarget{readonly} è un campo di sola lettura, dove non è possibile scrivere dati
	\item \gloTarget{RPC} è un  protocollo  che un programma può utilizzare per richiedere un servizio da un programma situato in un altro computer in una rete senza dover comprendere i dettagli della rete. Utilizza il   modello client-server. Il programma richiedente è un client e il servizio che fornisce il programma è il server.
	\item \gloTarget{slack} tempo aggiuntivo ad un'attività che ha lo scopo di evitare ritardi nella produzione del prodotto.
	\item \gloTarget{Time-to-Market} tempo che intercorre fra l’inizio del processo di sviluppo di un nuovo prodotto e l’avvio della sua commercializzazione.
	\item \gloTarget{Unità} Per unità si intende normalmente il minimo componente di un programma dotato di funzionamento autonomo e producibile da un singolo programmatore

\end{itemize} 
