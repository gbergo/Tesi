%!TEX root = ../dissertation.tex

\chapter{Introduzione}
\section{L'azienda}
\begin{figure}[H]
	\begin{center} \includegraphics[width=5cm]{figures/logo_synclab}
		\caption[Logo Sync Lab]{Logo Sync Lab} 
		\label{logo_python} 
	\end{center}
\end{figure}
Sync Lab è un'azienda che nasce come Software house tramutatasi rapidamente in una società di consulenza informatica attraverso un processo di maturazione delle competenze tecnologiche, e dal 2002 ad oggi ha raggiunto un organico aziendale di oltre 200 risorse.
L'azienda, propone sul mercato interessanti quanto innovativi prodotti software, nati nel proprio laboratorio di ricerca e sviluppo.\\
L'azienda ha sede legale a Napoli, ma ha anche altre sedi, come: Padova, Milano, Roma. 
Ha gradualmente conquistato significativamente fette di mercato nei seguenti settori: mobile, videosorveglianza e sicurezza delle infrastrutture informatiche aziendali.\\
Da questa realtà si evince come si debba essere sempre pronti a supportare il cliente nella realizzazione, messa in Opera e governance di soluzioni IT, sia dal punto di vista tecnologico, sia nel governo del cambiamento organizzativo.

\section{L'idea}
Il progetto che sta realizzando l'azienda, è in via di sviluppo e si basa sulla gestione delle risorse di un'azienda in ambito \glo{Food-and-beverage}.
L'idea è quella di mantenere ottimali i livelli di inventario riguardanti il \glo{ciclo di vita} degli alimenti e delle bevande, valutandoli tramite test per verificarne la correttezza del modulo ed il corretto funzionamento dello stesso.

\section{Organizzazione del testo}
Di seguito viene riportata per ogni capitolo una piccola descrizione delle tematiche trattate:
\begin{itemize}
	\item \hyperlink{(chap:capitolo2)}{\textbf{Capitolo 2}}: in questo capitolo vengono riportati gli obiettivi generali e la pianificazione concordata con l'azienda, inoltre vengono riportate le metodologie e strumenti utilizzati in generale, infine un'analisi dei rischi;
	\item \hyperlink{(chap:capitolo3)}{\textbf{Capitolo 3}}: viene descritto il problema generale, lo scopo dello stage e viene data una definizione del problema in esame;
	\item \hyperlink{(chap:capitolo4)}{\textbf{Capitolo 4}}: vengono riportati i moduli utilizzati e una descrizione delle loro caratteristiche, per ciascuno di essi verrà riportato integralmente la lista delle funzionalità odoo utilizzate, descrivendo inoltre anche cosa modellino; inoltre viene riportato per ciascun modello una descrizione dei vincoli, a corredo di tutto questo ci sarà del materiale grafico utilizzato durante lo stage;
	\item \hyperlink{(chap:capitolo5)}{\textbf{Capitolo 5}}: vengono riportate le modalità con cui si sono eseguiti i test, illustrando come siano stati strutturati e come sia stato possibile verificare le soluzioni fornite dai moduli;
	\item \hyperlink{(chap:capitolo6)}{\textbf{Capitolo 6}}: vengono riportati gli strumenti adottati per lo svolgimento delle attività, corredati da una breve descrizione che riporti come sono stati utilizzati;

	\item \hyperlink{(chap:capitolo8)}{\textbf{Capitolo 7}}: vengono riportate le conclusioni relative al numero di obiettivi soddisfatti e all'effettiva suddivisione delle ore rispetto tali obiettivi.

\end{itemize}
\vspace*{1cm}
Il testo adotta le seguenti convenzioni tipografiche:
\begin{itemize}
	\item ogni acronimo, abbreviazione, parola ambigua o tecnica viene spiegata e chiarificata alla fine del testo;
	\item ogni parola di glossario alla prima apparizione verrà etichetta come segue: $parola^{[g]}$;
	\item ogni riga di un elenco puntato terminerà con un ; a parte l'ultima riga che si concluderà con un punto.
\end{itemize}