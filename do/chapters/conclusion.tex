%!TEX root = ../dissertation.tex

\hypertarget{(chap:capitolo8)}{}
\chapter{Conclusioni}
Nel seguente capitolo verranno riportate le conclusioni che si è potuto trarre alla fine di questo stage.
\section{Consuntivo finale}
Prima di iniziare lo stage, si era concordato insieme con l'azienda la pianificazione dello stesso, definendo obiettivi da raggiungere e calendarizzandoli sulle 320 ore disponibili al fine di poterli soddisfare a pieno. La pianificazione iniziale è stata rispettata nella sua visione generale andando a modificarne invece le scadenze orarie come mostrato nella Tabella \ref{tab:consuntivo_finale}. La formazione è stata rispettata secondo il piano di lavoro, mentre nello sviluppo del primo modulo \bit{Odoo}{odoo} ho avuto qualche difficoltà. Questo perchè nella fase di formazione non si erano viste certe funzionalità come la creazione di un popup in caso di errore o la formazione di una template per creare l'e-mail. Questi problemi comunque sono stati pianamente risolti tramite un confronto con il tutor aziendale e i ragazzi in sede, che mi hanno fornito la giusta \bit{documentazione}{doc}, così da permettermi di completare tutti i moduli ed effettuare i relativi test che avevamo pianificato nel tempo previsto.

\begin{center}
	\begin{tabular}{|l|l|c l|}
		\hline
		\multicolumn{2}{|l|}{\textbf{Durata in ore}}		&	\multicolumn{2}{l|}{\textbf{Descrizione dell'attività}}\\
		\hline
		\multicolumn{2}{|l|}{180}	&	\multicolumn{2}{l|}{\textbf{A}: Formazione}\\
		\hline
		\multirow{5}{1cm}{ } & 8  & \hspace{5mm}•\hspace{2mm} & Studio Database Sql-PostgreSql \\
		\multirow{5}{1cm}{ } & 24 & \hspace{5mm}•\hspace{2mm} & Studio XML-XPath                 \\
		\multirow{5}{1cm}{ } & 4 & \hspace{5mm}•\hspace{2mm} & Studio Html5/CSS3\\
		\multirow{5}{1cm}{ } & 16 & \hspace{5mm}•\hspace{2mm} & Studio Javascript\\
		\multirow{5}{1cm}{ } & 40 & \hspace{5mm}•\hspace{2mm} & Studio linguaggio Python\\
		\multirow{5}{1cm}{ } & 32 & \hspace{5mm}•\hspace{2mm} & Studio Piattaforma Erp Odoo\\
		\multirow{5}{1cm}{ } & 56  & \hspace{5mm}•\hspace{2mm} & Sviluppo primo modulo di prova \\
		\hline																											
		\multicolumn{2}{|l|}{100}	&	\multicolumn{2}{l|}{\textbf{B}: Sviluppo moduli}\\
		\hline
		\multirow{5}{1cm}{ } & 100  & \hspace{5mm}•\hspace{2mm} & Implementazione di nuovi moduli Odoo  \\ & & & in ambito Food-and-beverage\\
		\hline
		
		\multicolumn{2}{|l|}{40}	&	\multicolumn{2}{l|}{\textbf{C}: Collaudo Finale}\\
		\hline
		\multirow{5}{1cm}{ } & 28  & \hspace{5mm}•\hspace{2mm} & Test\\
		\multirow{5}{1cm}{ } & 4 & \hspace{5mm}•\hspace{2mm} & Stesura Documentazione Finale \\
		\multirow{5}{1cm}{ } & 4 & \hspace{5mm}•\hspace{2mm} & Collaudo e consegna del codice\\
		
		\hline
		
		\multicolumn{2}{|l|}{\textbf{Totale: 320}} & \multicolumn{2}{l|}{}\\
		\hline
		
	\end{tabular}
	\captionof{table}{Ripartizione reale delle ore di stage}
	\label{tab:consuntivo_finale}  
\end{center}
\section{Raggiungimento degli obiettivi}
Gli obiettivi concordati prima dell'inizio dello stage e riportati nel \hyperlink{(chap:capitolo2)}{\textbf{Capitolo 2}}, prevedevano il soddisfacimento di quattro obiettivi obbligatori, due desiderabili e uno opzionale.
Gli obiettivi concordati sono stati soddisfatti a pieno tranne l'obbiettivo opzionale dove, visto il poco tempo rimasto si è deciso di eseguire i test di unità nei modelli dei moduli sviluppati, così da verificare al meglio il codice.\\
\newpage
\section{Conoscenze acquisite}
Le conoscenze acquisite durante il corso dello stage sono state le seguenti:
\begin{itemize}
	\item Linguaggio \bit{Python}{python};
	\item \bit{pgAdmin 4}{pgadmin};
	\item Librerie \bit{ORM Api}{orm};
	\item Costruzione modulo \bit{Odoo}{odoo};
	\item Package management attraverso \bit{Homebrew}{homebrew};
	\item Package management attraverso \bit{PIP}{pip};
	\item Approfondimento linguaggio XML/XPATH

\end{itemize}


\section{Sviluppi futuri del software}
Odoo è in continuo e rapido sviluppo. Esistono centinaia di moduli sviluppati dalla comunità e dalla casa madre liberamente utilizzabili, quindi a volte le cose che non riusciamo a trovare forse sono disponibili facendo le ricerche giuste. \\
I clienti per lavorare con Odoo non necessitano
di installare alcun software sul proprio pc ma usano un normale
programma per la navigazione internet su un computer collegato al
server, accedendo tramite inserimento di nome utente e password.
\'E un software basato sopratutto sulla fatturazione, la gestione dell'inventario ma come ho visto in azienda, è molto richiesto nella gestione di e-commerce. Lo sviluppo futuro sarà sicuramente quello di creare informazioni chiare e che trasmettono fiducia al cliente in fase di acquisto.


\newpage
\section{Valutazione personale}
La valutazione dello stage svolto presso Sync Lab non può che dirsi positiva, ho trovato un ambiente accogliente e dei colleghi sempre disponibili. Tecnicamente parlando affrontare un problema così complesso è stato formativo, poiché ho imparato ad usare strumenti utili ed un linguaggio molto popolare ed usato al giorno d'oggi.

I lunghi confronti con i colleghi sono stati importanti per capire l'approccio migliore per affrontare i vari task.\\
Lo stage è stato la conclusione di un percorso di studio molto intenso e credo si possa dire abbia messo un punto a tutti gli sforzi fatti e a mille pensieri per raggiungere l'obbiettivo.